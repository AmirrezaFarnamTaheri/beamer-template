% Set the document class to Beamer with desired options.
\documentclass[compress]{beamer}
% \documentclass[compress, aspectratio=169]{beamer} % For 16:9 aspect ratio
% \documentclass[compress, handout]{beamer} % For handout version

% ===========================================================
%               Beamer Theme and Color Settings
% ===========================================================

% ===========================
% === Outer Theme Options ===
% ===========================

% Set the outer theme.
% Available options:
% - 'shadow': adds subtle shadows around frames.
% - 'smoothbars': sleek, smooth navigation bars.
% - 'split': split frame appearance in the navigation area.
% - 'sidebar': places the navigation in a sidebar.
% - 'tree': hierarchical navigation.
% - 'miniframes': miniature frames as navigation symbols.
% - 'default': standard Beamer navigation.

% Define the outer theme here:
\newcommand{\outertheme}{shadow} % Change this to your desired outer theme.

% Use the selected outer theme.
\useoutertheme{\outertheme}

% If you want to pass options to the outer theme, you can define them here.
% For example, to disable subsections in miniframes:
% \useoutertheme[subsection=false]{miniframes}

% ===========================
% === Inner Theme Options ===
% ===========================

% Set the inner theme.
% Available options:
% - 'rounded': rounded corners for elements.
% - 'rectangles': rectangular elements with sharp corners.
% - 'inmargin': section titles in the margin.
% - 'circles': circular bullets in lists.

% Define the inner theme here:
\newcommand{\innertheme}{rounded} % Change this to your desired inner theme.

% Use the selected inner theme.
\useinnertheme[shadow=true]{\innertheme}

% ===========================
% === Color Theme Options ===
% ===========================
% Define custom color schemes using \definecolor and \setbeamercolor.
% Define color schemes with artistic choices
\definecolor{Rich_Red}{RGB}{164, 0, 0} % Rich Red for scheme 1
\definecolor{Light_Pink}{RGB}{255, 230, 230} % Light Pink for scheme 1

\definecolor{Deep_Blue}{RGB}{0, 70, 173} % Deep Blue for scheme 2
\definecolor{Light_Blue}{RGB}{220, 235, 255} % Light Blue for scheme 2

\definecolor{Teal}{RGB}{0, 128, 128} % Teal for scheme 3
\definecolor{Light_Cyan}{RGB}{224, 255, 255} % Light Cyan for scheme 3

\definecolor{Orange}{RGB}{255, 165, 0} % Orange for scheme 4
\definecolor{Light_Orange}{RGB}{255, 244, 224} % Light Orange for scheme 4

\definecolor{Forest_Green}{RGB}{34, 139, 34} % Forest Green for scheme 5
\definecolor{Light_Green}{RGB}{224, 255, 224} % Light Green for scheme 5

\definecolor{Steel_Blue}{RGB}{70, 130, 180} % Steel Blue for scheme 6
\definecolor{Light_Steel_Blue}{RGB}{235, 245, 255} % Light Steel Blue for scheme 6

\definecolor{Deep_Wine}{rgb}{0.5, 0.05, 0.20} % Deep Wine for scheme 7
\definecolor{White}{rgb}{1, 1, 1} % White for scheme 7

\definecolor{Crimson}{rgb}{0.66, 0.06, 0.24} % Crimson for scheme 8
\definecolor{White}{rgb}{1, 1, 1} % White for scheme 8

\definecolor{Indigo}{RGB}{75, 0, 130} % Indigo for scheme 9
\definecolor{Light_Lavender}{RGB}{240, 230, 255} % Light Lavender for scheme 9

\definecolor{Hot_Pink}{RGB}{255, 105, 180} % Hot Pink for scheme 10
\definecolor{Light_Pink}{RGB}{255, 230, 240} % Light Pink for scheme 10

\definecolor{Gold}{RGB}{255, 215, 0} % Gold for scheme 11
\definecolor{Light_Gold}{RGB}{255, 245, 224} % Light Gold for scheme 11

\definecolor{Purple}{RGB}{128, 0, 128} % Purple for scheme 12
\definecolor{Light_Purple}{RGB}{244, 224, 255} % Light Purple for scheme 12
% Define custom colors.
\definecolor{Rich_Red}{RGB}{164, 0, 0} % Rich Red
\definecolor{Light_Pink}{RGB}{255, 230, 230} % Light Pink
% (Other colors are defined similarly.)

% Define commands to set color schemes.
% Example for Color Scheme 1:
\newcommand{\setcolorschemeone}{
    \setbeamercolor{structure}{fg=Rich_Red} % Set structure color for scheme 1
    \setbeamercolor{titlelike}{parent=structure, bg=Rich_Red, fg=white} % Set title color for scheme 1
    \setbeamercolor{itemize item}{fg=Rich_Red} % Set item color for scheme 1
    \setbeamercolor{itemize subitem}{fg=Rich_Red} % Set subitem color for scheme 1
    \setbeamercolor{itemize subsubitem}{fg=Rich_Red} % Set subsubitem color for scheme 1
    \setbeamercolor{block title}{bg=Rich_Red, fg=white} % Set block title color for scheme 1
    \setbeamercolor{block body}{bg=Light_Pink, fg=black} % Set block body color for scheme 1
}

\newcommand{\setcolorschemetwo}{
    \setbeamercolor{structure}{fg=Deep_Blue} 
    \setbeamercolor{titlelike}{parent=structure, bg=Deep_Blue, fg=white} 
    \setbeamercolor{itemize item}{fg=Deep_Blue} 
    \setbeamercolor{itemize subitem}{fg=Deep_Blue} 
    \setbeamercolor{itemize subsubitem}{fg=Deep_Blue} 
    \setbeamercolor{block title}{bg=Deep_Blue, fg=white} 
    \setbeamercolor{block body}{bg=Light_Blue, fg=black} 
}

\newcommand{\setcolorschemethree}{
    \setbeamercolor{structure}{fg=Teal} 
    \setbeamercolor{titlelike}{parent=structure, bg=Teal, fg=white}
    \setbeamercolor{itemize item}{fg=Teal} 
    \setbeamercolor{itemize subitem}{fg=Teal} 
    \setbeamercolor{itemize subsubitem}{fg=Teal}
    \setbeamercolor{block title}{bg=Teal, fg=white}  
    \setbeamercolor{block body}{bg=Light_Cyan, fg=black} 
}

\newcommand{\setcolorschemefour}{
    \setbeamercolor{structure}{fg=Orange} 
    \setbeamercolor{titlelike}{parent=structure, bg=Orange, fg=white} 
    \setbeamercolor{itemize item}{fg=Orange} 
    \setbeamercolor{itemize subitem}{fg=Orange} 
    \setbeamercolor{itemize subsubitem}{fg=Orange} 
    \setbeamercolor{block title}{bg=Orange, fg=white} 
    \setbeamercolor{block body}{bg=Light_Orange, fg=black} 
}

\newcommand{\setcolorschemefive}{
    \setbeamercolor{structure}{fg=Forest_Green} 
    \setbeamercolor{titlelike}{parent=structure, bg=Forest_Green, fg=white} 
    \setbeamercolor{itemize item}{fg=Forest_Green} 
    \setbeamercolor{itemize subitem}{fg=Forest_Green} 
    \setbeamercolor{itemize subsubitem}{fg=Forest_Green} 
    \setbeamercolor{block title}{bg=Forest_Green, fg=white} 
    \setbeamercolor{block body}{bg=Light_Green, fg=black} 
}

\newcommand{\setcolorschemesix}{
    \setbeamercolor{structure}{fg=Steel_Blue} 
    \setbeamercolor{titlelike}{parent=structure, bg=Steel_Blue, fg=white} 
    \setbeamercolor{itemize item}{fg=Steel_Blue} 
    \setbeamercolor{itemize subitem}{fg=Steel_Blue} 
    \setbeamercolor{itemize subsubitem}{fg=Steel_Blue} 
    \setbeamercolor{block title}{bg=Steel_Blue, fg=white} 
    \setbeamercolor{block body}{bg=Light_Steel_Blue, fg=black} 
}

\newcommand{\setcolorschemeseven}{
    \setbeamercolor{structure}{fg=Deep_Wine} 
    \setbeamercolor{titlelike}{parent=structure, bg=Deep_Wine, fg=white} 
    \setbeamercolor{itemize item}{fg=Deep_Wine} 
    \setbeamercolor{itemize subitem}{fg=Deep_Wine} 
    \setbeamercolor{itemize subsubitem}{fg=Deep_Wine} 
    \setbeamercolor{block title}{bg=Deep_Wine, fg=white} 
    \setbeamercolor{block body}{bg=White, fg=black} 
}

\newcommand{\setcolorschemeeight}{
    \setbeamercolor{structure}{fg=Crimson} 
    \setbeamercolor{titlelike}{parent=structure, bg=Crimson, fg=white} 
    \setbeamercolor{itemize item}{fg=Crimson} 
    \setbeamercolor{itemize subitem}{fg=Crimson} 
    \setbeamercolor{itemize subsubitem}{fg=Crimson} 
    \setbeamercolor{block title}{bg=Crimson, fg=white} 
    \setbeamercolor{block body}{bg=White, fg=black} 
}

\newcommand{\setcolorschemenine}{
    \setbeamercolor{structure}{fg=Indigo} 
    \setbeamercolor{titlelike}{parent=structure, bg=Indigo, fg=white} 
    \setbeamercolor{itemize item}{fg=Indigo} 
    \setbeamercolor{itemize subitem}{fg=Indigo} 
    \setbeamercolor{itemize subsubitem}{fg=Indigo} 
    \setbeamercolor{block title}{bg=Indigo, fg=white} 
    \setbeamercolor{block body}{bg=Light_Lavender, fg=black} 
}

\newcommand{\setcolorschemeten}{
    \setbeamercolor{structure}{fg=Hot_Pink} 
    \setbeamercolor{titlelike}{parent=structure, bg=Hot_Pink, fg=white} 
    \setbeamercolor{itemize item}{fg=Hot_Pink} 
    \setbeamercolor{itemize subitem}{fg=Hot_Pink} 
    \setbeamercolor{itemize subsubitem}{fg=Hot_Pink} 
    \setbeamercolor{block title}{bg=Hot_Pink, fg=white} 
    \setbeamercolor{block body}{bg=Light_Pink, fg=black} 
}

\newcommand{\setcolorschemeeleven}{
    \setbeamercolor{structure}{fg=Gold} 
    \setbeamercolor{titlelike}{parent=structure, bg=Gold, fg=white} 
    \setbeamercolor{itemize item}{fg=Gold} 
    \setbeamercolor{itemize subitem}{fg=Gold} 
    \setbeamercolor{itemize subsubitem}{fg=Gold} 
    \setbeamercolor{block title}{bg=Gold, fg=white} 
    \setbeamercolor{block body}{bg=Light_Gold, fg=black} 
}

\newcommand{\setcolorschemetwelve}{
    \setbeamercolor{structure}{fg=Purple} 
    \setbeamercolor{titlelike}{parent=structure, bg=Purple, fg=white} 
    \setbeamercolor{itemize item}{fg=Purple} 
    \setbeamercolor{itemize subitem}{fg=Purple} 
    \setbeamercolor{itemize subsubitem}{fg=Purple} 
    \setbeamercolor{block title}{bg=Purple, fg=white} 
    \setbeamercolor{block body}{bg=Light_Purple, fg=black} 
}

% (Define other color schemes: \setcolorschemetwo, \setcolorschemethree, etc.)

% Choose the color scheme by setting \colorschemenumber to a value from 1 to 12.
\newcommand{\colorschemenumber}{7} % Change this number to select a different color scheme.

% Implement the selection of color schemes.
\ifcase\colorschemenumber
  \or % case 1
    \setcolorschemeone
  \or % case 2
    \setcolorschemetwo
  \or % case 3
    \setcolorschemethree
  \or % case 4
    \setcolorschemefour
  \or % case 5
    \setcolorschemefive
  \or % case 6
    \setcolorschemesix
  \or % case 7
    \setcolorschemeseven
  \or % case 8
    \setcolorschemeeight
  \or % case 9
    \setcolorschemenine
  \or % case 10
    \setcolorschemeten
  \or % case 11
    \setcolorschemeeleven
  \or % case 12
    \setcolorschemetwelve
\fi

% ===========================
% === Font Theme Options ===
% ===========================

% Set the font theme.
% Available options:
% - 'professionalfonts': uses the fonts specified in the document.
% - 'default': uses Beamer's default fonts.
% - 'serif': uses serif fonts.
% - 'structurebold': bolds structural elements.
% - 'sansserif': uses only sans-serif fonts.
% - 'serifmath': uses serif fonts in math mode.

% Define the font theme here:
\newcommand{\fonttheme}{professionalfonts} % Change this to your desired font theme.

% Use the selected font theme.
\usefonttheme{\fonttheme}

% Optionally, set specific fonts (requires XeLaTeX or LuaLaTeX).
% Uncomment the following lines if using XeLaTeX or LuaLaTeX.
% \usepackage{fontspec}
% \setmainfont{Times New Roman}
% \setsansfont{Arial}
% \setmonofont{Courier New}

% ===========================================================
%               Additional Package Imports
% ===========================================================

% Existing packages
\usepackage{etoolbox}
\usepackage{amsmath} % Enhanced math environment.
\usepackage{amssymb} % Additional math symbols.
\usepackage{mathtools} % Additional tools for mathematical typesetting.
\usepackage{graphicx} % Including graphics.
\usepackage{xcolor} % Colored text and boxes.
\usepackage{tikz} % Drawing graphics.
\usepackage{caption} % Standard captions.
\usepackage{subcaption} % Sub-captions within a figure.
\usepackage{listings} % Code listings.
\usepackage{hyperref} % Hyperlinks.
\usepackage[authoryear]{natbib} % Bibliographies and citations.
\usepackage{booktabs} % Improved table layouts.
\usepackage{multirow} % Control over tables.
\usepackage{array} % Additional table features.
\usepackage{siunitx} % Units and number formatting.
\usepackage[english]{babel} % Multilingual support.
\usepackage{cleveref} % Clever referencing.
\usepackage{setspace} % Adjusting line spacing.
\usepackage{tcolorbox} % Colored boxes.

% Additional packages required for the new features
\usepackage{appendixnumberbeamer} % For appendix numbering
\usepackage{ifthen} % For conditional statements in header/footer customization
% ===========================================================
%               Custom Header/Footer Options
% ===========================================================

% Define options to customize header and footer.
% You can set variables to include or exclude certain elements.

% Header options
\newcommand{\includeheader}{true} % Set to 'false' to remove the header.
\newcommand{\showsectionintoc}{true} % Set to 'false' to hide sections in the header.

% Footer options
\newcommand{\includefooter}{true} % Set to 'false' to remove the footer.
\newcommand{\showauthorinfo}{true} % Set to 'false' to hide author info in the footer.
\newcommand{\showdateinfo}{true} % Set to 'false' to hide date info in the footer.
\newcommand{\showpagenumber}{true} % Set to 'false' to hide page numbers in the footer.

% Customize the header
\ifthenelse{\equal{\includeheader}{true}}{
  \setbeamertemplate{headline}{
    % Customize the headline as desired
    \ifthenelse{\equal{\showsectionintoc}{true}}{
      % Include sections in the header
      \begin{beamercolorbox}[ht=2.5ex,dp=1ex,leftskip=1em,rightskip=1em]{section in head/foot}
        \usebeamerfont{section in head/foot}\textsc{\insertsectionnavigationhorizontal{\paperwidth}{}{}} % Use \textsc for small caps
      \end{beamercolorbox}
    }{
      % Do not include sections in the header
    }
  }
}{
  % Remove the header
  \setbeamertemplate{headline}{}
}

% Customize the footer
\ifthenelse{\equal{\includefooter}{true}}{
  \setbeamertemplate{footline}{
    \leavevmode%
    \hbox{%
      \ifthenelse{\equal{\showauthorinfo}{true}}{
        \begin{beamercolorbox}[wd=0.5\paperwidth,ht=2.25ex,dp=1ex,left]{author in head/foot}%
          \usebeamerfont{author in head/foot}\textsc{\insertshortauthor}~~(\textsc{\insertshortinstitute}) % Use \textsc for small caps
        \end{beamercolorbox}%
      }{}
      \begin{beamercolorbox}[wd=0.5\paperwidth,ht=2.25ex,dp=1ex,right]{date in head/foot}%
        \ifthenelse{\equal{\showdateinfo}{true}}{
          \usebeamerfont{date in head/foot}\textsc{\insertshortdate}\hspace*{2em} % Use \textsc for small caps
        }{}
        \ifthenelse{\equal{\showpagenumber}{true}}{
          \insertframenumber{} / \inserttotalframenumber\hspace*{2ex}
        }{}
      \end{beamercolorbox}}%
    \vskip0pt%
  }
}{
  % Remove the footer
  \setbeamertemplate{footline}{}
}

% ===========================================================
%       Enhanced Section Title Slide Template Options
% ===========================================================

% Define options for section starting slides
\newcommand{\sectiontitleslidebackground}{default} % Options: 'default', 'image', 'color'
\newcommand{\sectiontitleslideimage}{path/to/your/background/image.jpg} % Path to background image
\newcommand{\sectiontitleslidecolor}{blue} % Background color

% Configure a custom section title slide for when new sections begin.
\AtBeginSection[]{
  \logo{} % Hide the logo on the section title slide.

  % Prevent the section title slide from advancing the frame number.
  \addtocounter{framenumber}{-1}

  % Customize the section title slide background
  \ifthenelse{\equal{\sectiontitleslidebackground}{image}}{
    \setbeamertemplate{background}{
      \parbox[c][\paperheight][c]{\paperwidth}{%
        \vfill
        \centering
        \includegraphics[width=\paperwidth,height=\paperheight]{\sectiontitleslideimage}
        \vfill
      }
    }
  }{
    \ifthenelse{\equal{\sectiontitleslidebackground}{color}}{
      \setbeamercolor{background canvas}{bg=\sectiontitleslidecolor}
    }{
      % Use default background
    }
  }

  \begin{frame}[plain] % Use plain frame to remove header/footer
    \vfill
    \centering
    \begin{beamercolorbox}[sep=20pt, center, shadow=true, rounded=true]{title}
      \usebeamerfont{title}\textsc{\insertsectionhead}\par % Use \textsc for small caps
    \end{beamercolorbox}
    \vfill
  \end{frame}

  % Restore default background
  \setbeamertemplate{background}{}
  \setbeamercolor{background canvas}{bg=}

  % Restore the logo after the section title slide.
  % \logo{\includegraphics[height=0.4cm]{path/to/logo.png}}
}

% ===========================================================
%               Custom Command Definitions
% ===========================================================

% Command to highlight text with a yellow background.
\newcommand{\highlight}[1]{\colorbox{yellow!50}{#1}}

% Command to highlight notes with a pink background.
\newcommand{\notehighlight}[1]{\colorbox{pink!50}{#1}}

% Command to create a small plus sign using TikZ.
\newcommand{\Plus}{\mathord{\begin{tikzpicture}[baseline=0ex, line width=1, scale=0.13]
  \draw (1,0) -- (1,2);
  \draw (0,1) -- (2,1);
  \end{tikzpicture}}}

% Command for a custom colored box with text.
\newcommand{\custombox}[2]{\begin{tcolorbox}[colback=#1!5!white, colframe=#1!75!black]
#2
\end{tcolorbox}}

% Command for a colored alert text.
\newcommand{\alerttext}[1]{\textcolor{red}{\textbf{#1}}}

% Command for a custom footnote with symbol.
\newcommand{\symbolfootnote}[1]{\begingroup
\renewcommand{\thefootnote}{\fnsymbol{footnote}}
\footnote{#1}\endgroup}

% Command to include code snippets with syntax highlighting.
\lstset{
    basicstyle=\ttfamily\footnotesize,
    keywordstyle=\color{blue},
    commentstyle=\color{gray},
    stringstyle=\color{red},
    numbers=left,
    numberstyle=\tiny\color{gray},
    stepnumber=1,
    numbersep=5pt,
    breaklines=true,
    frame=single,
    language=Python % Change to the desired programming language.
}

% ===========================================================
%               Additional Presentation Settings
% ===========================================================

% Display navigation symbols (uncomment to show them).
% \setbeamertemplate{navigation symbols}{}

% Set the alignment of the frame title.
% Options: left, right, center.
% \setbeamertemplate{frametitle}[default][left]
% \setbeamertemplate{frametitle}[default][right]
% \setbeamertemplate{frametitle}[default][center]

% Include a custom logo on each slide.
% \logo{\includegraphics[height=1cm]{path/to/logo.png}}

% Set a custom image for itemize symbols.
% \setbeamertemplate{itemize item}{\includegraphics[height=0.3cm]{path/to/custom-bullet.png}}

% Set custom text margins.
\setbeamersize{text margin left=7mm, text margin right=7mm}

% Define the path to the folder containing images.
\graphicspath{{./images/}}

% Set line spacing for better readability.
\linespread{1.2} % Adjust as needed.

% ===========================================================
%            Title, Author, and Institute Information
% ===========================================================
% Title of the presentation.
% Use \title[Short Title]{Full Title}
\title[An Introduction to Machine Learning]{An Introduction to Machine Learning}

\subtitle{\textsc{By Authors \\ Journal Full Name, Year Published}}

% Author information.
% Use \author[Short Author]{Full Name\inst{1}}
\author[Farnam Taheri]{Amirreza "Farnam" Taheri\inst{1}}

% Institute information.
\institute[TeIAS]{\inst{1} Department of Economics, Tehran Institute for Advanced Studies}



% Add a logo to the title slide.
\titlegraphic{\includegraphics[height=0.8cm]{TEIAS-LOGO.png}}

% Date of the presentation.
\date[Nov 2024]{\small \today}

% ===========================================================
%              Begin Document
% ===========================================================

\begin{document}
% Remove navigation symbols for all slides.
\setbeamertemplate{navigation symbols}{}

% ===========================================================
%                Title Frame
% ===========================================================
% Re-enable headline and footline for following slides

\begin{frame}[plain, noframenumbering]
    \titlepage % Generates the title page.
\end{frame}

%  ==========================================
%              Motication frames
%  ==========================================

\begin{frame}{\textsc{Motivation}}
  \begin{itemize}
      \item Motivation
      \pause
      \vspace*{0.3cm}
      \item Continued
  \end{itemize}
\end{frame}


% ===========================================================
%                Outline Frame
% ===========================================================

\begin{frame}{\textsc{Outline}}
  \tableofcontents % Generates the table of contents.
\end{frame}

% ===========================================================
%             Section Title Slide Template
% ===========================================================

% (Already configured in the preamble.)

% ===========================================================
%             Content Frames
% ===========================================================

\section{Introduction}

\begin{frame}{\textsc{Introduction}}
  \begin{itemize}
    \item \textbf{What is Machine Learning?}
    \begin{itemize}
      \item A subset of artificial intelligence that provides systems the ability to automatically learn and improve from experience without being explicitly programmed.
    \end{itemize}
    \item \textbf{Why is it Important?}
    \begin{itemize}
      \item Enables computers to find hidden insights without being explicitly programmed where to look.
    \end{itemize}
    \item \textbf{Applications}
    \begin{itemize}
      \item Image recognition, speech recognition, medical diagnosis, stock market trading, etc.
    \end{itemize}
  \end{itemize}
\end{frame}

\section[Types]{Types of Machine Learning}

\begin{frame}{\textsc{Types of Machine Learning}}
  \begin{itemize}
    \item \textbf{Supervised Learning}
    \begin{itemize}
      \item The model is trained on labeled data.
      \item Examples: Regression, Classification.
    \end{itemize}
    \item \textbf{Unsupervised Learning}
    \begin{itemize}
      \item The model is trained on unlabeled data.
      \item Examples: Clustering, Association.
    \end{itemize}
    \item \textbf{Reinforcement Learning}
    \begin{itemize}
      \item The model learns through rewards and punishments.
      \item Examples: Game AI, Robotics.
    \end{itemize}
  \end{itemize}
\end{frame}

\section[Supervised]{Supervised Learning Algorithms}

\begin{frame}{\textsc{Supervised Learning Algorithms}}
  \begin{itemize}
    \item \textbf{Linear Regression}
    \item \textbf{Logistic Regression}
    \item \textbf{Decision Trees}
    \item \textbf{Support Vector Machines}
    \item \textbf{Neural Networks}
  \end{itemize}
\end{frame}

\section[Unsupervised]{Unsupervised Learning Algorithms}

\begin{frame}{\textsc{Unsupervised Learning Algorithms}}
  \begin{itemize}
    \item \textbf{K-Means Clustering}
    \item \textbf{Hierarchical Clustering}
    \item \textbf{Principal Component Analysis (PCA)}
    \item \textbf{Anomaly Detection}
  \end{itemize}
\end{frame}

\section[Applications]{Applications of Machine Learning}

\begin{frame}{\textsc{Applications of Machine Learning}}
  \begin{itemize}
    \item \textbf{Healthcare}
    \begin{itemize}
      \item Disease prediction, personalized medicine.
    \end{itemize}
    \item \textbf{Finance}
    \begin{itemize}
      \item Fraud detection, algorithmic trading.
    \end{itemize}
    \item \textbf{Transportation}
    \begin{itemize}
      \item Autonomous vehicles, traffic prediction.
    \end{itemize}
    \item \textbf{Retail}
    \begin{itemize}
      \item Customer segmentation, demand forecasting.
    \end{itemize}
  \end{itemize}
\end{frame}

\section{Conclusion}

\begin{frame}{\textsc{Conclusion}}
  \begin{itemize}
    \item Machine learning is transforming industries by enabling data-driven decisions.
    \item Understanding the basics is essential for leveraging its full potential.
    \item Continuous learning and adaptation are key in this rapidly evolving field.
  \end{itemize}
\end{frame}

% ===========================================================
%            Sample Frames with Different Content
% ===========================================================

% Frame with highlighted content.
\begin{frame}{\textsc{Key Takeaways}}
  \highlight{Machine Learning} is a powerful tool for extracting insights from data.

  \begin{itemize}
    \item It is essential to choose the right algorithm for the task.
    \item Data quality significantly impacts model performance.
    \item Ethical considerations are crucial when applying machine learning.
  \end{itemize}

  \notehighlight{Note:} Always validate your models with real-world data.
\end{frame}

% Frame with a colored box using the custom command.
\begin{frame}{\textsc{Important Considerations}}
  \custombox{Rich_Red}{
    \begin{itemize}
      \item \textbf{Bias and Fairness:} Ensure your models do not perpetuate biases.
      \item \textbf{Privacy:} Protect sensitive data and comply with regulations.
      \item \textbf{Interpretability:} Strive for models that are explainable.
    \end{itemize}
  }
\end{frame}

% Frame with mathematical equations.
\begin{frame}{\textsc{Mathematical Foundations}}
  \textbf{Linear Regression Model:}
  \begin{equation}
    y = \beta_0 + \beta_1 x_1 + \beta_2 x_2 + \dots + \beta_n x_n + \epsilon
  \end{equation}
  \begin{itemize}
    \item $y$: Dependent variable
    \item $x_i$: Independent variables
    \item $\beta_i$: Coefficients
    \item $\epsilon$: Error term
  \end{itemize}

  \textbf{Cost Function:}
  \begin{equation}
    J(\theta) = \frac{1}{2m} \sum_{i=1}^{m} (h_{\theta}(x^{(i)}) - y^{(i)})^2
  \end{equation}
  \begin{itemize}
    \item $J(\theta)$: Cost function
    \item $h_{\theta}(x)$: Hypothesis function
    \item $m$: Number of training examples
  \end{itemize}
\end{frame}

% Frame with code listing.
\begin{frame}[fragile]{\textsc{Code Example: Linear Regression in Python}}
  \begin{lstlisting}[language=Python]
import numpy as np
from sklearn.linear_model import LinearRegression

# Sample data
X = np.array([[1, 1], [1, 2], [2, 2], [2, 3]])
y = np.dot(X, np.array([1, 2])) + 3

# Create model and fit
model = LinearRegression().fit(X, y)

# Predictions
predictions = model.predict(X)
print(predictions)
  \end{lstlisting}
\end{frame}

% % Frame with an image.
% \begin{frame}{\textsc{Machine Learning Workflow}}
%   \begin{figure}
%     \centering
%     \includegraphics[width=0.8\textwidth]{ml_workflow.png}
%     \caption{Typical steps in a machine learning project.}
%   \end{figure}
% \end{frame}

% Frame with a table.
\begin{frame}{\textsc{Comparison of Algorithms}}
  \begin{table}[h!]
    \centering
    \begin{tabular}{lccc}
      \toprule
      \textbf{Algorithm} & \textbf{Type} & \textbf{Complexity} & \textbf{Interpretability} \\
      \midrule
      Linear Regression & Regression & Low & High \\
      Decision Trees & Both & Medium & Medium \\
      Neural Networks & Both & High & Low \\
      \bottomrule
    \end{tabular}
    \caption{Comparison of common machine learning algorithms.}
  \end{table}
\end{frame}

% % Frame with multiple images.
% \begin{frame}{\textsc{Visualization Examples}}
%   \begin{figure}
%     \centering
%     \begin{subfigure}{0.45\textwidth}
%       \centering
%       \includegraphics[width=\textwidth]{scatter_plot.png}
%       \caption{Scatter Plot}
%     \end{subfigure}
%     \hfill
%     \begin{subfigure}{0.45\textwidth}
%       \centering
%       \includegraphics[width=\textwidth]{confusion_matrix.png}
%       \caption{Confusion Matrix}
%     \end{subfigure}
%     \caption{Common visualizations in machine learning.}
%   \end{figure}
% \end{frame}

% ===========================================================
%             Thank You Slide
% ===========================================================

\section*{} % Empty section for Thank You slide.

\begin{frame}[plain, noframenumbering]
  \vfill
  \centering
  \textcolor{Rich_Red}{\Huge\scshape{Thank You!}}
  \vfill

  % Insert the logo as a figure
  \begin{figure}[h!]
    \includegraphics[height=1.5cm]{TEIAS-LOGO}
  \end{figure}

% Re-enable headline and footline for following slides
\setbeamertemplate{headline}[miniframes theme]
\setbeamertemplate{footline}[miniframes theme]

% Remove title graphic
\titlegraphic{}
\end{frame}

% ===========================================================
%             References Slide
% ===========================================================
\begin{frame}[allowframebreaks]{\textsc{References}}
  \bibliographystyle{abbrvnat}
  \bibliography{references} % Make sure to create a 'references.bib' file.
\end{frame}

% ===========================================================
%             Appendix Section
% ===========================================================

% Start the appendix
\appendix

\section*{Appendix}

\begin{frame}{\textsc{Additional Resources}}
  \begin{itemize}
    \item Books:
    \begin{itemize}
      \item \emph{Pattern Recognition and Machine Learning} by Christopher M. Bishop
      \item \emph{Machine Learning: A Probabilistic Perspective} by Kevin P. Murphy
    \end{itemize}
    \item Online Courses:
    \begin{itemize}
      \item Coursera: Machine Learning by Andrew Ng
      \item edX: Introduction to Artificial Intelligence (AI)
    \end{itemize}
  \end{itemize}
\end{frame}

\begin{frame}{\textsc{Mathematical Background}}
  \begin{itemize}
    \item \textbf{Probability and Statistics}
    \item \textbf{Linear Algebra}
    \item \textbf{Calculus}
    \item \textbf{Optimization Techniques}
  \end{itemize}
\end{frame}

% ===========================================================
%              End of Document
% ===========================================================

\end{document}
